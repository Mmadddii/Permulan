\nonstopmode{}
\documentclass[a4paper]{book}
\usepackage[times,inconsolata,hyper]{Rd}
\usepackage{makeidx}
\makeatletter\@ifl@t@r\fmtversion{2018/04/01}{}{\usepackage[utf8]{inputenc}}\makeatother
% \usepackage{graphicx} % @USE GRAPHICX@
\makeindex{}
\begin{document}
\chapter*{}
\begin{center}
{\textbf{\huge Package `Permulan'}}
\par\bigskip{\large \today}
\end{center}
\ifthenelse{\boolean{Rd@use@hyper}}{\hypersetup{pdftitle = {Permulan: Datuen aurreprozesamendurako eta azterketa estatistikorako tresna multzoa (S4 egiturak erabiliz)}}}{}
\ifthenelse{\boolean{Rd@use@hyper}}{\hypersetup{pdfauthor = {Maddi Olaetxea}}}{}
\begin{description}
\raggedright{}
\item[Type]\AsIs{Package}
\item[Title]\AsIs{Datuen aurreprozesamendurako eta azterketa estatistikorako
tresna multzoa (S4 egiturak erabiliz)}
\item[Version]\AsIs{0.1.0}
\item[Description]\AsIs{Pakete honek datuen aurreprozesamendurako eta azterketa estatistikorako tresna multzoa eskaintzen du,
S4 klaseen bidez egituratuta. Hezkuntza helburuetarako erabilera sustatzen du eta adibide praktikoak eskaintzen ditu.}
\item[License]\AsIs{MIT + file LICENSE}
\item[Encoding]\AsIs{UTF-8}
\item[RoxygenNote]\AsIs{7.3.3}
\item[Imports]\AsIs{ggplot2, reshape2, scales, methods}
\item[Suggests]\AsIs{knitr, rmarkdown}
\item[VignetteBuilder]\AsIs{knitr}
\item[NeedsCompilation]\AsIs{no}
\item[Author]\AsIs{Maddi Olaetxea [aut, cre]}
\item[Maintainer]\AsIs{Maddi Olaetxea }\email{maddiolausa@gmail.com}\AsIs{}
\end{description}
\Rdcontents{Contents}
\HeaderA{Permulan-package}{Permulan: Datuen aurreprozesamendurako eta azterketa estatistikorako tresna multzoa (S4 egiturak erabiliz)}{Permulan.Rdash.package}
\aliasA{Permulan}{Permulan-package}{Permulan}
\keyword{internal}{Permulan-package}
%
\begin{Description}
Pakete honek datuen aurreprozesamendurako eta azterketa estatistikorako tresna multzoa eskaintzen du, S4 klaseen bidez egituratuta. Hezkuntza helburuetarako erabilera sustatzen du eta adibide praktikoak eskaintzen ditu.
\end{Description}
%
\begin{Author}
\strong{Maintainer}: Maddi Olaetxea \email{maddiolausa@gmail.com}

\end{Author}
\HeaderA{Atributua}{Atributua instantzia sortzeko funtzioa}{Atributua}
%
\begin{Description}
Atributu bat sortzen du, bere izena, mota eta balioekin.
\end{Description}
%
\begin{Usage}
\begin{verbatim}
Atributua(izena, balioak, mota = NULL)
\end{verbatim}
\end{Usage}
%
\begin{Arguments}
\begin{ldescription}
\item[\code{izena}] Atributuaren izena (karaktere kate bat).

\item[\code{balioak}] Atributuaren balioak (zenbakizkoak edo kategorikoak).

\item[\code{mota}] Atributuaren mota (aukerazkoa: `"numeric"` edo `"categoric"`).
\end{ldescription}
\end{Arguments}
%
\begin{Value}
"Atributua" klaseko objektu bat.
\end{Value}
%
\begin{Examples}
\begin{ExampleCode}
Atributua("adina", c(10, 20, 30))
\end{ExampleCode}
\end{Examples}
\HeaderA{Atributua-class}{Atributua klasea}{Atributua.Rdash.class}
%
\begin{Description}
Datuen atributu baten egitura eta bere propietateak azaltzen ditu.
\end{Description}
%
\begin{Section}{Slots}

\begin{description}

\item[\code{izena}] Atributuaren izena (karaktere kate bat).

\item[\code{mota}] Atributuaren mota: `"numeric"` edo `"categoric"`.

\item[\code{balioak}] Atributuaren balioak (numeric, factor edo bestelakoak).

\end{description}
\end{Section}
%
\begin{Examples}
\begin{ExampleCode}
a <- Atributua("adina", c(20, 30, 40, 50, 60, 70, 80))
a

\end{ExampleCode}
\end{Examples}
\HeaderA{DatuMultzoa}{DatuMultzoa instantzia sortzeko funtzioa}{DatuMultzoa}
%
\begin{Description}
'DatuMultzoa' klaseko objektu bat sortzen du `data.frame` batetik.
\end{Description}
%
\begin{Usage}
\begin{verbatim}
DatuMultzoa(datuak, klase_izena = NULL)
\end{verbatim}
\end{Usage}
%
\begin{Arguments}
\begin{ldescription}
\item[\code{datuak}] Data.frame bat atributuak eta klase aldagaia jasotzen dituena.

\item[\code{klase\_izena}] Klase gisa erabili beharreko zutabearen izena (aukerakoa).
\end{ldescription}
\end{Arguments}
%
\begin{Value}
"DatuMultzoa" klaseko objektu bat.
\end{Value}
%
\begin{Examples}
\begin{ExampleCode}
datuak <- data.frame(adina = c(23, 45, 34), sexua = c("M", "F", "M"))
DatuMultzoa(datuak, klase_izena = "sexua")
\end{ExampleCode}
\end{Examples}
\HeaderA{DatuMultzoa-class}{DatuMultzoa S4 klasearen definizioa}{DatuMultzoa.Rdash.class}
%
\begin{Description}
'DatuMultzoa' klaseak atributu multzo bat eta klase-aldagai bat gordetzen ditu.
\end{Description}
%
\begin{Section}{Slots}

\begin{description}

\item[\code{atributuak}] Atributuen zerrenda bat (`Atributua` objektuen lista).

\item[\code{klasea}] Klase aldagaia (`Atributua` klaseko objektu bat).

\end{description}
\end{Section}
%
\begin{SeeAlso}
[Atributua] klasea, atributu indibidualak definitzeko.
\end{SeeAlso}
\HeaderA{discretize}{Diskretizazioa puntu ebakiekin}{discretize}
\aliasA{discretize,Atributua-method}{discretize}{discretize,Atributua.Rdash.method}
\aliasA{discretize,DatuMultzoa-method}{discretize}{discretize,DatuMultzoa.Rdash.method}
\aliasA{discretize,numeric-method}{discretize}{discretize,numeric.Rdash.method}
%
\begin{Description}
`discretize()` funtzioak balio jarraituak kategoria tarteetan bihurtzen ditu
emandako ebaki puntuak erabiliz.
Funtzio honek `numeric`, `Atributua` eta `DatuMultzoa` klaseetarako metodoak ditu.
\end{Description}
%
\begin{Usage}
\begin{verbatim}
discretize(x, cut.points)

## S4 method for signature 'numeric'
discretize(x, cut.points)

## S4 method for signature 'Atributua'
discretize(x, cut.points)

## S4 method for signature 'DatuMultzoa'
discretize(x, cut.points)
\end{verbatim}
\end{Usage}
%
\begin{Arguments}
\begin{ldescription}
\item[\code{x}] Diskretizatzeko objektua (`numeric`, `Atributua` edo `DatuMultzoa`)

\item[\code{cut.points}] Ebaki puntuak adierazten dituen bektore zenbakizkoa
\end{ldescription}
\end{Arguments}
%
\begin{Value}
\begin{itemize}

\item{} `numeric` sarreran: diskretizatutako balioak eta ebaki puntuak dituen zerrenda bat.
\item{} `Atributua` sarreran: atributu kategoriko bihurtutako `Atributua` objektu bat.
\item{} `DatuMultzoa` sarreran: atributu zenbakizko guztiak diskretizatutako `DatuMultzoa` objektu bat.

\end{itemize}

\end{Value}
%
\begin{Examples}
\begin{ExampleCode}
balioak <- c(20, 30, 40, 50)
discretize(balioak, c(25, 35, 45))

\end{ExampleCode}
\end{Examples}
\HeaderA{discretizeEF}{Equal Frequency diskretizazioa}{discretizeEF}
\aliasA{discretizeEF,Atributua-method}{discretizeEF}{discretizeEF,Atributua.Rdash.method}
\aliasA{discretizeEF,numeric-method}{discretizeEF}{discretizeEF,numeric.Rdash.method}
%
\begin{Description}
`discretizeEF()` funtzioak balio jarraituak kuantilen arabera
tarte berdinetan banatzen ditu (maiztasun berdineko binak).
Funtzio honek `numeric` eta `Atributua` klaseetarako metodoak ditu.
\end{Description}
%
\begin{Usage}
\begin{verbatim}
discretizeEF(x, num.bins)

## S4 method for signature 'numeric'
discretizeEF(x, num.bins)

## S4 method for signature 'Atributua'
discretizeEF(x, num.bins)
\end{verbatim}
\end{Usage}
%
\begin{Arguments}
\begin{ldescription}
\item[\code{x}] Diskretizatzeko objektua (`numeric` edo `Atributua`)

\item[\code{num.bins}] Zenbat tarte sortu nahi diren adierazten duen balio zenbakizkoa
\end{ldescription}
\end{Arguments}
%
\begin{Value}
\begin{itemize}

\item{} `numeric` sarreran: diskretizatutako balioak eta ebaki puntuak dituen zerrenda bat.
\item{} `Atributua` sarreran: kategoria tarteetan bihurtutako `Atributua` objektua eta bere ebaki puntuak.

\end{itemize}

\end{Value}
%
\begin{Examples}
\begin{ExampleCode}
x <- c(10, 20, 25, 30, 40, 50, 60, 70)
discretizeEF(x, num.bins = 3)

\end{ExampleCode}
\end{Examples}
\HeaderA{discretizeEW}{Equal Width diskretizazioa}{discretizeEW}
\aliasA{discretizeEW,Atributua-method}{discretizeEW}{discretizeEW,Atributua.Rdash.method}
\aliasA{discretizeEW,numeric-method}{discretizeEW}{discretizeEW,numeric.Rdash.method}
%
\begin{Description}
`discretizeEW()` funtzioak balio jarraituak tarte zabalera berdinetan banatzen ditu.
Hau da, datuen balio minimo eta maximoaren artean `num.bins` tarte sortzen dira,
non tarte bakoitzaren zabalera konstantea den.

Funtzio honek `numeric` eta `Atributua` klaseetarako metodoak ditu.
\end{Description}
%
\begin{Usage}
\begin{verbatim}
discretizeEW(x, num.bins)

## S4 method for signature 'numeric'
discretizeEW(x, num.bins)

## S4 method for signature 'Atributua'
discretizeEW(x, num.bins)
\end{verbatim}
\end{Usage}
%
\begin{Arguments}
\begin{ldescription}
\item[\code{x}] Diskretizatzeko datuak (`numeric` edo `Atributua` klaseko objektua)

\item[\code{num.bins}] Sortu nahi diren tarte kopurua (zenbakizkoa, >1)
\end{ldescription}
\end{Arguments}
%
\begin{Value}
\begin{itemize}

\item{} `numeric` sarreran: diskretizatutako balioak eta ebaki puntuak dituen zerrenda.
\item{} `Atributua` sarreran: kategoriatan bihurtutako `Atributua` objektua eta bere ebaki puntuak.

\end{itemize}

\end{Value}
%
\begin{Examples}
\begin{ExampleCode}
# --- Adibidea 1: numeric bektore batekin ---
x <- c(10, 20, 25, 30, 40, 50, 60, 70)
emaitza <- discretizeEW(x, num.bins = 3)
emaitza$x.discretized
emaitza$cut.points

# --- Adibidea 2: Atributua klasearekin ---
a <- Atributua("adina", c(20, 30, 40, 50, 60, 70, 80))
a_disk <- discretizeEW(a, num.bins = 3)
a_disk$atributua@balioak
a_disk$cut.points

\end{ExampleCode}
\end{Examples}
\HeaderA{entropy}{Entropiaren kalkulua atributu kategorikoentzat}{entropy}
\aliasA{entropy,Atributua-method}{entropy}{entropy,Atributua.Rdash.method}
\aliasA{entropy,factor-method}{entropy}{entropy,factor.Rdash.method}
\aliasA{entropy\_by\_column}{entropy}{entropy.Rul.by.Rul.column}
\aliasA{entropy\_by\_column,data.frame-method}{entropy}{entropy.Rul.by.Rul.column,data.frame.Rdash.method}
\aliasA{entropy\_by\_column,DatuMultzoa-method}{entropy}{entropy.Rul.by.Rul.column,DatuMultzoa.Rdash.method}
%
\begin{Description}
`entropy()` funtzioak aldagai diskretu baten entropia kalkulatzen du.
`entropy\_by\_column()` funtzioak dataset edo `DatuMultzoa` batean
atributu guztien entropia kalkulatzen du zutabeka.
\end{Description}
%
\begin{Usage}
\begin{verbatim}
entropy(x)

## S4 method for signature 'factor'
entropy(x)

## S4 method for signature 'Atributua'
entropy(x)

entropy_by_column(x)

## S4 method for signature 'data.frame'
entropy_by_column(x)

## S4 method for signature 'DatuMultzoa'
entropy_by_column(x)
\end{verbatim}
\end{Usage}
%
\begin{Arguments}
\begin{ldescription}
\item[\code{x}] Objektua (`factor`, `Atributua`, `data.frame` edo `DatuMultzoa`).
\end{ldescription}
\end{Arguments}
%
\begin{Value}
`entropy()` → balio bakarra (`numeric`),
`entropy\_by\_column()` → bektore izenduna (`numeric`).
\end{Value}
%
\begin{SeeAlso}
`var\_col()` eta `roc\_analisi()`
\end{SeeAlso}
%
\begin{Examples}
\begin{ExampleCode}
fakt <- factor(c("A", "A", "B", "B", "C", "A"))
entropy(fakt)
df <- data.frame(kol1 = factor(c("A", "A", "B", "C")),
                 kol2 = factor(c("X", "X", "X", "Y")))
entropy_by_column(df)
dm <- DatuMultzoa(df)
entropy_by_column(dm)

\end{ExampleCode}
\end{Examples}
\HeaderA{filter\_metrics}{Atributuen filtraketa metriketan oinarrituta}{filter.Rul.metrics}
\aliasA{filter\_metrics,data.frame-method}{filter\_metrics}{filter.Rul.metrics,data.frame.Rdash.method}
\aliasA{filter\_metrics,DatuMultzoa-method}{filter\_metrics}{filter.Rul.metrics,DatuMultzoa.Rdash.method}
%
\begin{Description}
Funtzio hauek dataset baten atributuak filtratzen dituzte
aurrez kalkulatutako metrikak kontuan hartuta:
\begin{itemize}

\item{} Bariantza minimoa atributu numerikoentzat
\item{} Entropia minimoa atributu kategorikoentzat
\item{} AUC minimoa atributu numerikoentzat (klasea emanda)

\end{itemize}


Metodoak `data.frame` eta `DatuMultzoa` klaseetarako definituta daude.
\end{Description}
%
\begin{Usage}
\begin{verbatim}
filter_metrics(x, ...)

## S4 method for signature 'data.frame'
filter_metrics(x, var_min = 0, entropy_min = 0, auc_min = 0, klase = NULL, ...)

## S4 method for signature 'DatuMultzoa'
filter_metrics(x, var_min = 0, entropy_min = 0, auc_min = 0, ...)
\end{verbatim}
\end{Usage}
%
\begin{Arguments}
\begin{ldescription}
\item[\code{x}] Dataset bat (`data.frame` edo `DatuMultzoa`)

\item[\code{...}] Argumentu gehigarriak; metodoen arteko bateragarritasunerako.

\item[\code{var\_min}] Bariantza minimoa (lehenetsia = 0)

\item[\code{entropy\_min}] Entropia minimoa (lehenetsia = 0)

\item[\code{auc\_min}] AUC minimoa (lehenetsia = 0)

\item[\code{klase}] Klase aldagaia edo bektore binarioa AUC kalkulatzeko (`data.frame` metodoan bakarrik erabiltzen da)
\end{ldescription}
\end{Arguments}
%
\begin{Value}
Filtratutako objektu berria (`data.frame` edo `DatuMultzoa`)
\end{Value}
%
\begin{SeeAlso}
`var\_col()` — atributu numerikoen bariantza kalkulatzeko,
`entropy()` — atributu kategorikoen entropia kalkulatzeko,
`roc\_analisi()` — AUC kalkulatzeko.
\end{SeeAlso}
%
\begin{Examples}
\begin{ExampleCode}
df <- data.frame(
  x1 = c(1,2,3,4,5),
  x2 = c(5,5,5,5,5),
  cat = factor(c("A","B","A","B","C")),
  label = c(TRUE, FALSE, TRUE, FALSE, TRUE)
)

# Data.frame baten adibidea
filter_metrics(df, var_min = 1, entropy_min = 0.5)

# DatuMultzoa objektu baten adibidea
dm <- DatuMultzoa(df, klase_izena = "label")
filter_metrics(dm, var_min = 1, entropy_min = 0.5)

\end{ExampleCode}
\end{Examples}
\HeaderA{Korrelazioak\_irudikatu}{Korrelazioen irudikapena}{Korrelazioak.Rul.irudikatu}
%
\begin{Description}
Dataset baten korrelazio-matrizea edo informazio mutuala irudikatzen du `ggplot2` erabiliz.
Zenbakizko aldagaietarako korrelazioak kalkulatzen dira.
*Adi!!!* Jatorrizko datuak sartu behar dira, funtzioak falkulatzen ditu korrelazioak eta bertatik irudikatzen ditu.
\end{Description}
%
\begin{Usage}
\begin{verbatim}
Korrelazioak_irudikatu(M)
\end{verbatim}
\end{Usage}
%
\begin{Arguments}
\begin{ldescription}
\item[\code{M}] Data.frame edo matrize bat, non zutabe guztiak zenbakizkoak diren.
\end{ldescription}
\end{Arguments}
%
\begin{Value}
Korrelazio-matrizearen heatmap bat bistaratzen du.
\end{Value}
%
\begin{Examples}
\begin{ExampleCode}
df <- data.frame(
  adina = c(20, 25, 30, 35, 40, 45, 50),
  soldata = c(5000, 4000, 3000, 2000, 1000, 800, 600),
  altuera = c(1.60, 1.65, 1.70, 1.75, 1.78, 1.82, 1.85)
)
Korrelazioak_irudikatu(df)

\end{ExampleCode}
\end{Examples}
\HeaderA{korrelazio\_matrizea}{Korrelazioak edo informazio mutua aldagai guztien artean}{korrelazio.Rul.matrizea}
\aliasA{korrelazio\_matrizea,data.frame-method}{korrelazio\_matrizea}{korrelazio.Rul.matrizea,data.frame.Rdash.method}
\aliasA{korrelazio\_matrizea,DatuMultzoa-method}{korrelazio\_matrizea}{korrelazio.Rul.matrizea,DatuMultzoa.Rdash.method}
%
\begin{Description}
Datu-multzo bateko aldagai guztien arteko erlazioa kalkulatzen du:
\begin{itemize}

\item{} Pearson korrelazioa (biak zenbakizkoak direnean)
\item{} Informazio mutuoa (biak kategorikoak direnean)

\end{itemize}

Funtzioak aldagai bakoitzaren mota automatikoki detektatzen du.
\end{Description}
%
\begin{Usage}
\begin{verbatim}
korrelazio_matrizea(x, ...)

## S4 method for signature 'data.frame'
korrelazio_matrizea(x, ...)

## S4 method for signature 'DatuMultzoa'
korrelazio_matrizea(x, ...)
\end{verbatim}
\end{Usage}
%
\begin{Arguments}
\begin{ldescription}
\item[\code{x}] Data frame edo \code{DatuMultzoa} objektua.

\item[\code{...}] Argumentu gehigarriak, erabilera barnekoetarako.
\end{ldescription}
\end{Arguments}
%
\begin{Value}
Korrelazio edo informazio mutua matrize bat.
\end{Value}
%
\begin{Examples}
\begin{ExampleCode}
df <- data.frame(
  adina = c(20, 25, 30, 35, 40),
  soldata = c(1000, 1200, 1500, 1800, 2000),
  sexua = factor(c("M", "F", "M", "M", "F")),
  hiria = factor(c("A", "B", "A", "C", "B"))
)

korrelazio_matrizea(df)

\end{ExampleCode}
\end{Examples}
\HeaderA{normalize}{Datuen normalizazioa eta estandarizazioa}{normalize}
\aliasA{normalize,Atributua-method}{normalize}{normalize,Atributua.Rdash.method}
\aliasA{normalize,data.frame-method}{normalize}{normalize,data.frame.Rdash.method}
\aliasA{normalize,DatuMultzoa-method}{normalize}{normalize,DatuMultzoa.Rdash.method}
\aliasA{normalize,numeric-method}{normalize}{normalize,numeric.Rdash.method}
\aliasA{standardize}{normalize}{standardize}
\aliasA{standardize,Atributua-method}{normalize}{standardize,Atributua.Rdash.method}
\aliasA{standardize,data.frame-method}{normalize}{standardize,data.frame.Rdash.method}
\aliasA{standardize,DatuMultzoa-method}{normalize}{standardize,DatuMultzoa.Rdash.method}
\aliasA{standardize,numeric-method}{normalize}{standardize,numeric.Rdash.method}
%
\begin{Description}
Bi funtzio generiko eskaintzen dira:
\begin{itemize}

\item{} `normalize()`: datuak 0 eta 1 arteko tartean eskalatzen ditu.
\item{} `standardize()`: datuak batezbesteko 0 eta desbideratze estandar 1 duten eskalara eramaten ditu.

\end{itemize}


Metodoak `numeric`, `Atributua`, `data.frame` eta `DatuMultzoa` klaseetarako definituta daude.
\end{Description}
%
\begin{Usage}
\begin{verbatim}
normalize(x)

## S4 method for signature 'numeric'
normalize(x)

## S4 method for signature 'numeric'
standardize(x)

## S4 method for signature 'Atributua'
normalize(x)

## S4 method for signature 'Atributua'
standardize(x)

## S4 method for signature 'DatuMultzoa'
normalize(x)

## S4 method for signature 'DatuMultzoa'
standardize(x)

## S4 method for signature 'data.frame'
normalize(x)

## S4 method for signature 'data.frame'
standardize(x)
\end{verbatim}
\end{Usage}
%
\begin{Arguments}
\begin{ldescription}
\item[\code{x}] Normalizatu edo estandarizatu beharreko objektua.
\end{ldescription}
\end{Arguments}
%
\begin{Value}
Objektu bera, baina balioak normalizatuak edo estandarizatuak:
\begin{itemize}

\item{} `numeric`: bektore zenbakizkoa
\item{} `Atributua`: `Atributua` objektu berria
\item{} `data.frame`: datu-markoa eskalatua
\item{} `DatuMultzoa`: atributuak eskalatu dituen objektua

\end{itemize}

\end{Value}
%
\begin{SeeAlso}
`var\_col()` — bariantza kalkulatzeko,
`entropy()` — atributu kategorikoen informazio maila neurtzeko.
\end{SeeAlso}
%
\begin{Examples}
\begin{ExampleCode}
x <- c(10, 20, 30)
normalize(x)
standardize(x)

df <- data.frame(a = 1:5, b = 6:10)
normalize(df)
standardize(df)

\end{ExampleCode}
\end{Examples}
\HeaderA{roc\_analisi}{AUC kalkulua atributu jarraituentzat}{roc.Rul.analisi}
\aliasA{roc\_analisi,data.frame-method}{roc\_analisi}{roc.Rul.analisi,data.frame.Rdash.method}
\aliasA{roc\_analisi,DatuMultzoa-method}{roc\_analisi}{roc.Rul.analisi,DatuMultzoa.Rdash.method}
%
\begin{Description}
`roc\_analisi()` funtzioak atributu jarraitu baten eta klase binario baten arteko
erlazioa ebaluatzen du, ROC kurba eta AUC balioa kalkulatuz.
\end{Description}
%
\begin{Usage}
\begin{verbatim}
roc_analisi(df)

## S4 method for signature 'data.frame'
roc_analisi(df)

## S4 method for signature 'DatuMultzoa'
roc_analisi(df)
\end{verbatim}
\end{Usage}
%
\begin{Arguments}
\begin{ldescription}
\item[\code{df}] `data.frame` edo `DatuMultzoa` objektua.
\end{ldescription}
\end{Arguments}
%
\begin{Value}
Zerrenda bat: `curva` (ROC kurbaren puntu guztiak) eta `AUC` (azalera).
\end{Value}
%
\begin{SeeAlso}
`var\_col()` eta `entropy()`
\end{SeeAlso}
%
\begin{Examples}
\begin{ExampleCode}
df_test <- data.frame(
  aldagaia = c(0.1, 0.4, 0.35, 0.8, 0.9, 0.2, 0.6, 0.5, 0.7, 0.3),
  etiketa  = c(FALSE, FALSE, TRUE, TRUE, TRUE, FALSE, TRUE, TRUE, TRUE, FALSE)
)
emaitza <- roc_analisi(df_test)
print(emaitza$AUC)

\end{ExampleCode}
\end{Examples}
\HeaderA{show,DatuMultzoa-method}{DatuMultzoa objektua erakusteko metodoa}{show,DatuMultzoa.Rdash.method}
%
\begin{Description}
`show()` metodoa 'DatuMultzoa' klaseko objektuak pantailaratzeko.
\end{Description}
%
\begin{Usage}
\begin{verbatim}
## S4 method for signature 'DatuMultzoa'
show(object)
\end{verbatim}
\end{Usage}
%
\begin{Arguments}
\begin{ldescription}
\item[\code{object}] DatuMultzoa klaseko objektua.
\end{ldescription}
\end{Arguments}
\HeaderA{var\_col}{Atributu zenbakizkoen bariantza kalkulua}{var.Rul.col}
\aliasA{var\_col,Atributua-method}{var\_col}{var.Rul.col,Atributua.Rdash.method}
\aliasA{var\_col,data.frame-method}{var\_col}{var.Rul.col,data.frame.Rdash.method}
\aliasA{var\_col,DatuMultzoa-method}{var\_col}{var.Rul.col,DatuMultzoa.Rdash.method}
%
\begin{Description}
`var\_col()` funtzioak atributu zenbakizkoen bariantza kalkulatzen du.
Metodoak `data.frame`, `Atributua` eta `DatuMultzoa` klaseetarako bertsioak ditu.
\end{Description}
%
\begin{Usage}
\begin{verbatim}
var_col(x)

## S4 method for signature 'data.frame'
var_col(x)

## S4 method for signature 'DatuMultzoa'
var_col(x)

## S4 method for signature 'Atributua'
var_col(x)
\end{verbatim}
\end{Usage}
%
\begin{Arguments}
\begin{ldescription}
\item[\code{x}] Aztertzeko objektua (`data.frame`, `Atributua` edo `DatuMultzoa`).
\end{ldescription}
\end{Arguments}
%
\begin{Value}
Bariantza balioak dituen bektore izenduna edo balio bakarra (`numeric`).
\end{Value}
%
\begin{SeeAlso}
`entropy()` — atributu kategorikoen informazio maila kalkulatzeko.
\end{SeeAlso}
%
\begin{Examples}
\begin{ExampleCode}
df <- data.frame(a = 1:5, b = c(2,3,4,5,6))
var_col(df)

\end{ExampleCode}
\end{Examples}
\HeaderA{Visualizar\_ROC}{ROC kurbaren irudikapena (AUC plot)}{Visualizar.Rul.ROC}
%
\begin{Description}
`roc\_analisi()` funtzioaren emaitzetatik ROC kurba marrazten du eta
AUC balioa irudikatzen du.
\end{Description}
%
\begin{Usage}
\begin{verbatim}
Visualizar_ROC(resultado)
\end{verbatim}
\end{Usage}
%
\begin{Arguments}
\begin{ldescription}
\item[\code{resultado}] `roc\_analisi()` funtzioaren emaitza (`curva` eta `AUC` elementuak dituen zerrenda)
\end{ldescription}
\end{Arguments}
%
\begin{Value}
ROC kurba marrazten du pantailan.
\end{Value}
%
\begin{Examples}
\begin{ExampleCode}
df <- data.frame(
  adina = c(20, 25, 30, 35, 40, 45, 50),
  soldata = c(5000, 4000, 3000, 2000, 1000, 800, 600),
  altuera = c(1.60, 1.65, 1.70, 1.75, 1.78, 1.82, 1.85)
)
Korrelazioak_irudikatu(df)

\end{ExampleCode}
\end{Examples}
\printindex{}
\end{document}
